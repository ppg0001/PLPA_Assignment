% Options for packages loaded elsewhere
\PassOptionsToPackage{unicode}{hyperref}
\PassOptionsToPackage{hyphens}{url}
%
\documentclass[
]{article}
\usepackage{amsmath,amssymb}
\usepackage{iftex}
\ifPDFTeX
  \usepackage[T1]{fontenc}
  \usepackage[utf8]{inputenc}
  \usepackage{textcomp} % provide euro and other symbols
\else % if luatex or xetex
  \usepackage{unicode-math} % this also loads fontspec
  \defaultfontfeatures{Scale=MatchLowercase}
  \defaultfontfeatures[\rmfamily]{Ligatures=TeX,Scale=1}
\fi
\usepackage{lmodern}
\ifPDFTeX\else
  % xetex/luatex font selection
\fi
% Use upquote if available, for straight quotes in verbatim environments
\IfFileExists{upquote.sty}{\usepackage{upquote}}{}
\IfFileExists{microtype.sty}{% use microtype if available
  \usepackage[]{microtype}
  \UseMicrotypeSet[protrusion]{basicmath} % disable protrusion for tt fonts
}{}
\makeatletter
\@ifundefined{KOMAClassName}{% if non-KOMA class
  \IfFileExists{parskip.sty}{%
    \usepackage{parskip}
  }{% else
    \setlength{\parindent}{0pt}
    \setlength{\parskip}{6pt plus 2pt minus 1pt}}
}{% if KOMA class
  \KOMAoptions{parskip=half}}
\makeatother
\usepackage{xcolor}
\usepackage[margin=1in]{geometry}
\usepackage{color}
\usepackage{fancyvrb}
\newcommand{\VerbBar}{|}
\newcommand{\VERB}{\Verb[commandchars=\\\{\}]}
\DefineVerbatimEnvironment{Highlighting}{Verbatim}{commandchars=\\\{\}}
% Add ',fontsize=\small' for more characters per line
\usepackage{framed}
\definecolor{shadecolor}{RGB}{248,248,248}
\newenvironment{Shaded}{\begin{snugshade}}{\end{snugshade}}
\newcommand{\AlertTok}[1]{\textcolor[rgb]{0.94,0.16,0.16}{#1}}
\newcommand{\AnnotationTok}[1]{\textcolor[rgb]{0.56,0.35,0.01}{\textbf{\textit{#1}}}}
\newcommand{\AttributeTok}[1]{\textcolor[rgb]{0.13,0.29,0.53}{#1}}
\newcommand{\BaseNTok}[1]{\textcolor[rgb]{0.00,0.00,0.81}{#1}}
\newcommand{\BuiltInTok}[1]{#1}
\newcommand{\CharTok}[1]{\textcolor[rgb]{0.31,0.60,0.02}{#1}}
\newcommand{\CommentTok}[1]{\textcolor[rgb]{0.56,0.35,0.01}{\textit{#1}}}
\newcommand{\CommentVarTok}[1]{\textcolor[rgb]{0.56,0.35,0.01}{\textbf{\textit{#1}}}}
\newcommand{\ConstantTok}[1]{\textcolor[rgb]{0.56,0.35,0.01}{#1}}
\newcommand{\ControlFlowTok}[1]{\textcolor[rgb]{0.13,0.29,0.53}{\textbf{#1}}}
\newcommand{\DataTypeTok}[1]{\textcolor[rgb]{0.13,0.29,0.53}{#1}}
\newcommand{\DecValTok}[1]{\textcolor[rgb]{0.00,0.00,0.81}{#1}}
\newcommand{\DocumentationTok}[1]{\textcolor[rgb]{0.56,0.35,0.01}{\textbf{\textit{#1}}}}
\newcommand{\ErrorTok}[1]{\textcolor[rgb]{0.64,0.00,0.00}{\textbf{#1}}}
\newcommand{\ExtensionTok}[1]{#1}
\newcommand{\FloatTok}[1]{\textcolor[rgb]{0.00,0.00,0.81}{#1}}
\newcommand{\FunctionTok}[1]{\textcolor[rgb]{0.13,0.29,0.53}{\textbf{#1}}}
\newcommand{\ImportTok}[1]{#1}
\newcommand{\InformationTok}[1]{\textcolor[rgb]{0.56,0.35,0.01}{\textbf{\textit{#1}}}}
\newcommand{\KeywordTok}[1]{\textcolor[rgb]{0.13,0.29,0.53}{\textbf{#1}}}
\newcommand{\NormalTok}[1]{#1}
\newcommand{\OperatorTok}[1]{\textcolor[rgb]{0.81,0.36,0.00}{\textbf{#1}}}
\newcommand{\OtherTok}[1]{\textcolor[rgb]{0.56,0.35,0.01}{#1}}
\newcommand{\PreprocessorTok}[1]{\textcolor[rgb]{0.56,0.35,0.01}{\textit{#1}}}
\newcommand{\RegionMarkerTok}[1]{#1}
\newcommand{\SpecialCharTok}[1]{\textcolor[rgb]{0.81,0.36,0.00}{\textbf{#1}}}
\newcommand{\SpecialStringTok}[1]{\textcolor[rgb]{0.31,0.60,0.02}{#1}}
\newcommand{\StringTok}[1]{\textcolor[rgb]{0.31,0.60,0.02}{#1}}
\newcommand{\VariableTok}[1]{\textcolor[rgb]{0.00,0.00,0.00}{#1}}
\newcommand{\VerbatimStringTok}[1]{\textcolor[rgb]{0.31,0.60,0.02}{#1}}
\newcommand{\WarningTok}[1]{\textcolor[rgb]{0.56,0.35,0.01}{\textbf{\textit{#1}}}}
\usepackage{graphicx}
\makeatletter
\def\maxwidth{\ifdim\Gin@nat@width>\linewidth\linewidth\else\Gin@nat@width\fi}
\def\maxheight{\ifdim\Gin@nat@height>\textheight\textheight\else\Gin@nat@height\fi}
\makeatother
% Scale images if necessary, so that they will not overflow the page
% margins by default, and it is still possible to overwrite the defaults
% using explicit options in \includegraphics[width, height, ...]{}
\setkeys{Gin}{width=\maxwidth,height=\maxheight,keepaspectratio}
% Set default figure placement to htbp
\makeatletter
\def\fps@figure{htbp}
\makeatother
\setlength{\emergencystretch}{3em} % prevent overfull lines
\providecommand{\tightlist}{%
  \setlength{\itemsep}{0pt}\setlength{\parskip}{0pt}}
\setcounter{secnumdepth}{-\maxdimen} % remove section numbering
\ifLuaTeX
  \usepackage{selnolig}  % disable illegal ligatures
\fi
\usepackage{bookmark}
\IfFileExists{xurl.sty}{\usepackage{xurl}}{} % add URL line breaks if available
\urlstyle{same}
\hypersetup{
  pdftitle={Coding\_Challenge4\_Markdown},
  pdfauthor={Pankaj Gaonkar},
  hidelinks,
  pdfcreator={LaTeX via pandoc}}

\title{Coding\_Challenge4\_Markdown}
\author{Pankaj Gaonkar}
\date{2025-02-27}

\begin{document}
\maketitle

\section{Coding Challenge 4 Answers}\label{coding-challenge-4-answers}

Q1) Explain the following a. YAML header YAMLS helps in writing the
configuration of the files and is on the top of the markdown files.

\begin{enumerate}
\def\labelenumi{\alph{enumi}.}
\setcounter{enumi}{1}
\tightlist
\item
  Literate programming In literate programming along with our code we
  will have its explanations and provide the chunks of codes. Itexplains
  the codes in bettwe way.
\end{enumerate}

Q2)

\subsection{Below is the clickable link to manuscript where these data
are
published}\label{below-is-the-clickable-link-to-manuscript-where-these-data-are-published}

\href{https://doi.org/10.1094/PDIS-06-21-1253-RE}{Noel et al., 2022.
Endophytic fungi as promising biocontrol agent to protect wheat from
Fusarium graminearum head blight. Plant Disease}

\begin{Shaded}
\begin{Highlighting}[]
\DocumentationTok{\#\# Load libraries}
\FunctionTok{library}\NormalTok{(tidyverse)}
\end{Highlighting}
\end{Shaded}

\begin{verbatim}
## -- Attaching core tidyverse packages ------------------------ tidyverse 2.0.0 --
## v dplyr     1.1.4     v readr     2.1.5
## v forcats   1.0.0     v stringr   1.5.1
## v ggplot2   3.5.1     v tibble    3.2.1
## v lubridate 1.9.3     v tidyr     1.3.1
## v purrr     1.0.2     
## -- Conflicts ------------------------------------------ tidyverse_conflicts() --
## x dplyr::filter() masks stats::filter()
## x dplyr::lag()    masks stats::lag()
## i Use the conflicted package (<http://conflicted.r-lib.org/>) to force all conflicts to become errors
\end{verbatim}

\begin{Shaded}
\begin{Highlighting}[]
\FunctionTok{library}\NormalTok{(ggpubr)}
\FunctionTok{library}\NormalTok{(}\StringTok{"knitr"}\NormalTok{)  }\CommentTok{\# required for knitting}


\DocumentationTok{\#\# Reading data by reative path}

\NormalTok{csv }\OtherTok{\textless{}{-}} \FunctionTok{read.csv}\NormalTok{(}\StringTok{"MycotoxinData.csv"}\NormalTok{, }\AttributeTok{na.strings =} \StringTok{"na"}\NormalTok{)}
\FunctionTok{head}\NormalTok{ (csv)}
\end{Highlighting}
\end{Shaded}

\begin{verbatim}
##   Treatment Cultivar BioRep MassperSeed_mg   DON X15ADON
## 1        Fg  Wheaton      2      10.291304 107.3    3.00
## 2        Fg  Wheaton      2      12.803226  32.6    0.85
## 3        Fg  Wheaton      2       2.846667 416.0    3.50
## 4        Fg  Wheaton      2       6.500000 211.9    3.10
## 5        Fg  Wheaton      2      10.179167 124.0    4.80
## 6        Fg  Wheaton      2      12.044444  73.1    3.30
\end{verbatim}

\begin{Shaded}
\begin{Highlighting}[]
\DocumentationTok{\#\# Codes needed for complete analysis}

\CommentTok{\#Plot1 Treatment on Y}

\NormalTok{P1 }\OtherTok{\textless{}{-}} \FunctionTok{ggplot}\NormalTok{(csv, }\FunctionTok{aes}\NormalTok{(}\AttributeTok{x =}\NormalTok{ Treatment, }\AttributeTok{y =}\NormalTok{ DON, }\AttributeTok{fill =}\NormalTok{ Cultivar))}\SpecialCharTok{+}
  \FunctionTok{theme\_classic}\NormalTok{() }\SpecialCharTok{+}                \CommentTok{\# removes the grids}
  \FunctionTok{geom\_boxplot}\NormalTok{() }\SpecialCharTok{+} 
  \FunctionTok{xlab}\NormalTok{(}\StringTok{""}\NormalTok{) }\SpecialCharTok{+}
  \FunctionTok{ylab}\NormalTok{(}\StringTok{"DON (ppm)"}\NormalTok{) }\SpecialCharTok{+}
  \FunctionTok{geom\_point}\NormalTok{(}\AttributeTok{alpha =} \FloatTok{0.6}\NormalTok{, }\AttributeTok{shape =} \DecValTok{21}\NormalTok{, }\AttributeTok{position =} \FunctionTok{position\_jitterdodge}\NormalTok{()) }\SpecialCharTok{+}
  \FunctionTok{scale\_fill\_manual}\NormalTok{ (}\AttributeTok{values =} \FunctionTok{c}\NormalTok{(}\StringTok{"\#56B4E9"}\NormalTok{, }\StringTok{"\#009E73"}\NormalTok{)) }\SpecialCharTok{+}
  \FunctionTok{facet\_wrap}\NormalTok{(}\SpecialCharTok{\textasciitilde{}}\NormalTok{ Cultivar) }

\FunctionTok{str}\NormalTok{(csv)}
\end{Highlighting}
\end{Shaded}

\begin{verbatim}
## 'data.frame':    375 obs. of  6 variables:
##  $ Treatment     : chr  "Fg" "Fg" "Fg" "Fg" ...
##  $ Cultivar      : chr  "Wheaton" "Wheaton" "Wheaton" "Wheaton" ...
##  $ BioRep        : int  2 2 2 2 2 2 2 2 2 3 ...
##  $ MassperSeed_mg: num  10.29 12.8 2.85 6.5 10.18 ...
##  $ DON           : num  107.3 32.6 416 211.9 124 ...
##  $ X15ADON       : num  3 0.85 3.5 3.1 4.8 3.3 6.9 2.9 2.1 0.71 ...
\end{verbatim}

\begin{Shaded}
\begin{Highlighting}[]
\NormalTok{csv}\SpecialCharTok{$}\NormalTok{Treatment }\OtherTok{\textless{}{-}} \FunctionTok{factor}\NormalTok{(csv}\SpecialCharTok{$}\NormalTok{Treatment, }\AttributeTok{levels =} \FunctionTok{c}\NormalTok{(}\StringTok{"NTC"}\NormalTok{, }\StringTok{"Fg"}\NormalTok{,}\StringTok{"Fg + 37"}\NormalTok{, }\StringTok{"Fg + 40"}\NormalTok{, }\StringTok{"Fg + 70"}\NormalTok{))}


\CommentTok{\#Plot2 X15ADON on Y}

\NormalTok{P2 }\OtherTok{\textless{}{-}} \FunctionTok{ggplot}\NormalTok{(csv, }\FunctionTok{aes}\NormalTok{(}\AttributeTok{x =}\NormalTok{ Treatment, }\AttributeTok{y =}\NormalTok{ X15ADON, }\AttributeTok{fill =}\NormalTok{ Cultivar))}\SpecialCharTok{+}
  \FunctionTok{theme\_classic}\NormalTok{() }\SpecialCharTok{+}                \CommentTok{\# removes the grids}
  \FunctionTok{geom\_boxplot}\NormalTok{() }\SpecialCharTok{+} 
  \FunctionTok{xlab}\NormalTok{(}\StringTok{""}\NormalTok{) }\SpecialCharTok{+}
  \FunctionTok{ylab}\NormalTok{(}\StringTok{"15ADON"}\NormalTok{) }\SpecialCharTok{+}
  \FunctionTok{geom\_point}\NormalTok{(}\AttributeTok{alpha =} \FloatTok{0.6}\NormalTok{, }\AttributeTok{shape =} \DecValTok{21}\NormalTok{, }\AttributeTok{position =} \FunctionTok{position\_jitterdodge}\NormalTok{()) }\SpecialCharTok{+}
  \FunctionTok{scale\_fill\_manual}\NormalTok{ (}\AttributeTok{values =} \FunctionTok{c}\NormalTok{(}\StringTok{"\#56B4E9"}\NormalTok{, }\StringTok{"\#009E73"}\NormalTok{)) }\SpecialCharTok{+}
  \FunctionTok{facet\_wrap}\NormalTok{(}\SpecialCharTok{\textasciitilde{}}\NormalTok{ Cultivar) }


\CommentTok{\#Plot3 MassperSeed\_mg on Y}

\NormalTok{P3 }\OtherTok{\textless{}{-}} \FunctionTok{ggplot}\NormalTok{(csv, }\FunctionTok{aes}\NormalTok{(}\AttributeTok{x =}\NormalTok{ Treatment, }\AttributeTok{y =}\NormalTok{ MassperSeed\_mg, }\AttributeTok{fill =}\NormalTok{ Cultivar))}\SpecialCharTok{+}
  \FunctionTok{theme\_classic}\NormalTok{() }\SpecialCharTok{+}                \CommentTok{\# removes the grids}
  \FunctionTok{geom\_boxplot}\NormalTok{() }\SpecialCharTok{+} 
  \FunctionTok{xlab}\NormalTok{(}\StringTok{""}\NormalTok{) }\SpecialCharTok{+}
  \FunctionTok{ylab}\NormalTok{(}\StringTok{"Seed Mass (mg)"}\NormalTok{) }\SpecialCharTok{+}
  \FunctionTok{geom\_point}\NormalTok{(}\AttributeTok{alpha =} \FloatTok{0.6}\NormalTok{, }\AttributeTok{shape =} \DecValTok{21}\NormalTok{, }\AttributeTok{position =} \FunctionTok{position\_jitterdodge}\NormalTok{()) }\SpecialCharTok{+}
  \FunctionTok{scale\_fill\_manual}\NormalTok{ (}\AttributeTok{values =} \FunctionTok{c}\NormalTok{(}\StringTok{"\#56B4E9"}\NormalTok{, }\StringTok{"\#009E73"}\NormalTok{)) }\SpecialCharTok{+}
  \FunctionTok{facet\_wrap}\NormalTok{(}\SpecialCharTok{\textasciitilde{}}\NormalTok{ Cultivar)}
\end{Highlighting}
\end{Shaded}

\subsection{Code from coding challenge 3, question 5 \& sepreated code
chunks for each
plit}\label{code-from-coding-challenge-3-question-5-sepreated-code-chunks-for-each-plit}

\begin{Shaded}
\begin{Highlighting}[]
\NormalTok{P1\_stat }\OtherTok{\textless{}{-}}\NormalTok{ P1 }\SpecialCharTok{+} 
            \FunctionTok{geom\_pwc}\NormalTok{(}\FunctionTok{aes}\NormalTok{(}\AttributeTok{group =}\NormalTok{ Treatment), }\AttributeTok{method =} \StringTok{"t\_test"}\NormalTok{, }\AttributeTok{label =} \StringTok{"p.adj.format"}\NormalTok{)}
\NormalTok{P1\_stat}
\end{Highlighting}
\end{Shaded}

\begin{verbatim}
## Warning: Removed 8 rows containing non-finite outside the scale range
## (`stat_boxplot()`).
\end{verbatim}

\begin{verbatim}
## Warning: Removed 8 rows containing non-finite outside the scale range
## (`stat_pwc()`).
\end{verbatim}

\begin{verbatim}
## Warning: Removed 8 rows containing missing values or values outside the scale range
## (`geom_point()`).
\end{verbatim}

\includegraphics{Coding_challenge_markdown_files/figure-latex/P1-1.pdf}

\begin{Shaded}
\begin{Highlighting}[]
\NormalTok{P2\_stat }\OtherTok{\textless{}{-}}\NormalTok{ P2 }\SpecialCharTok{+} 
  \FunctionTok{geom\_pwc}\NormalTok{(}\FunctionTok{aes}\NormalTok{(}\AttributeTok{group =}\NormalTok{ Treatment), }\AttributeTok{method =} \StringTok{"t\_test"}\NormalTok{, }\AttributeTok{label =} \StringTok{"p.adj.format"}\NormalTok{)}
\NormalTok{P2\_stat}
\end{Highlighting}
\end{Shaded}

\begin{verbatim}
## Warning: Removed 10 rows containing non-finite outside the scale range
## (`stat_boxplot()`).
\end{verbatim}

\begin{verbatim}
## Warning: Removed 10 rows containing non-finite outside the scale range
## (`stat_pwc()`).
\end{verbatim}

\begin{verbatim}
## Warning: Removed 10 rows containing missing values or values outside the scale range
## (`geom_point()`).
\end{verbatim}

\includegraphics{Coding_challenge_markdown_files/figure-latex/P2-1.pdf}

\begin{Shaded}
\begin{Highlighting}[]
\NormalTok{P3\_stat }\OtherTok{\textless{}{-}}\NormalTok{ P3 }\SpecialCharTok{+} 
  \FunctionTok{geom\_pwc}\NormalTok{(}\FunctionTok{aes}\NormalTok{(}\AttributeTok{group =}\NormalTok{ Treatment), }\AttributeTok{method =} \StringTok{"t\_test"}\NormalTok{, }\AttributeTok{label =} \StringTok{"p.adj.format"}\NormalTok{)}
\NormalTok{P3\_stat}
\end{Highlighting}
\end{Shaded}

\begin{verbatim}
## Warning: Removed 2 rows containing non-finite outside the scale range
## (`stat_boxplot()`).
\end{verbatim}

\begin{verbatim}
## Warning: Removed 2 rows containing non-finite outside the scale range
## (`stat_pwc()`).
\end{verbatim}

\begin{verbatim}
## Warning: Removed 2 rows containing missing values or values outside the scale range
## (`geom_point()`).
\end{verbatim}

\includegraphics{Coding_challenge_markdown_files/figure-latex/P3-1.pdf}

\begin{Shaded}
\begin{Highlighting}[]
\CommentTok{\#combining}
\NormalTok{Combined\_plot\_stat }\OtherTok{\textless{}{-}} \FunctionTok{ggarrange}\NormalTok{(}
\NormalTok{  P1\_stat,   }\DocumentationTok{\#\# add the plots onf interest subsequently}
\NormalTok{  P2\_stat,}
\NormalTok{  P3\_stat,}
  \AttributeTok{labels =} \StringTok{"auto"}\NormalTok{,  }\CommentTok{\# Automatically label the plots (A, B, C, etc.)}
  \AttributeTok{nrow =} \DecValTok{1}\NormalTok{,  }\CommentTok{\# Arrange the plots in \# rows}
  \AttributeTok{ncol =} \DecValTok{3}\NormalTok{,  }\CommentTok{\# Arrange the plots in \# column}
  \AttributeTok{common.legend =}\NormalTok{ T}
\NormalTok{)}
\end{Highlighting}
\end{Shaded}

\begin{verbatim}
## Warning: Removed 8 rows containing non-finite outside the scale range
## (`stat_boxplot()`).
\end{verbatim}

\begin{verbatim}
## Warning: Removed 8 rows containing non-finite outside the scale range
## (`stat_pwc()`).
\end{verbatim}

\begin{verbatim}
## Warning: Removed 8 rows containing missing values or values outside the scale range
## (`geom_point()`).
\end{verbatim}

\begin{verbatim}
## Warning: Removed 8 rows containing non-finite outside the scale range
## (`stat_boxplot()`).
\end{verbatim}

\begin{verbatim}
## Warning: Removed 8 rows containing non-finite outside the scale range
## (`stat_pwc()`).
\end{verbatim}

\begin{verbatim}
## Warning: Removed 8 rows containing missing values or values outside the scale range
## (`geom_point()`).
\end{verbatim}

\begin{verbatim}
## Warning: Removed 10 rows containing non-finite outside the scale range
## (`stat_boxplot()`).
\end{verbatim}

\begin{verbatim}
## Warning: Removed 10 rows containing non-finite outside the scale range
## (`stat_pwc()`).
\end{verbatim}

\begin{verbatim}
## Warning: Removed 10 rows containing missing values or values outside the scale range
## (`geom_point()`).
\end{verbatim}

\begin{verbatim}
## Warning: Removed 2 rows containing non-finite outside the scale range
## (`stat_boxplot()`).
\end{verbatim}

\begin{verbatim}
## Warning: Removed 2 rows containing non-finite outside the scale range
## (`stat_pwc()`).
\end{verbatim}

\begin{verbatim}
## Warning: Removed 2 rows containing missing values or values outside the scale range
## (`geom_point()`).
\end{verbatim}

\subsubsection{GitHub link:}\label{github-link}

\url{https://github.com/ppg0001/PLPA_Assignment/tree/main}

\end{document}
